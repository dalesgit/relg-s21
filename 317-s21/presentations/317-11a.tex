\begin{document}

\maketitle
\begin{frame}{Outline}
\setcounter{tocdepth}{1}
\tableofcontents
\end{frame}

\section{5 Themes}
\label{sec-1}
\begin{frame}[label=sec-1-1]{Humanity / Divinity of Christ}
\begin{itemize}
\item John 1  (NRSV) 1 In the beginning was the Word, and the Word was with God, and the Word was God. 2 He was in the beginning with God. 3 All things came into being through him, and without him not one thing came into being. What has come into being 4 in him was life, \ldots{}
\item Matthew 13:55 Is not this the carpenter’s son? Is not his mother called Mary? And are not his brothers James and Joseph and Simon and Judas?
\end{itemize}
\end{frame}

\begin{frame}[label=sec-1-2]{Spirit and Structure}
\begin{itemize}
\item Romans 8:2 For the law of the Spirit of life in Christ Jesus has set you free from the law of sin and of death.
\end{itemize}
\end{frame}

\begin{frame}[label=sec-1-3]{Reason and Revelation}
\begin{itemize}
\item Romans 16:25  Now to God who is able to strengthen you according to my gospel and the proclamation of Jesus Christ, according to the revelation of the mystery that was kept secret for long ages
\end{itemize}
\end{frame}

\begin{frame}[label=sec-1-4]{Works and Grace}
\begin{itemize}
\item 2 Timothy 1:9 who saved us and called us with a holy calling, not according to our works but according to his own purpose and grace. This grace was given to us in Christ Jesus before the ages began,
\end{itemize}
\end{frame}

\begin{frame}[label=sec-1-5]{Church and State}
\begin{itemize}
\item Romans 13 Let every person be subject to the governing authorities; for there is no authority except from God, and those authorities that exist have been instituted by God.
\end{itemize}
\end{frame}

\section{Chapter 17}
\label{sec-2}
\begin{frame}[label=sec-2-1]{Trust in History, progress, rational.}
The 19th c. began with a reaction against the perceived sterile rationalism of the
Enlightenment and a trust that History itself was moving forward, upward, and in a better
direction. Progress was inevitably being made through a kind of invisible Hegelian god­like
hand. What evidence do you see, for and against the progress of history?
\end{frame}

\begin{frame}[label=sec-2-2]{``Genius''}
The text notes that ``A new spirit of nationalism created new interest in various national
traditions ­­ the brothers Grimm scoured the countryside for old folktales, composers
incorporated folk tunes in their symphonies.'' The previous chapter was titled ``City on a Hill''
and seemed to focus on the particular ``genius'' of the American experiment. Discuss the
pros and cons of the ``new spirit of nationalism'' referenced in the 17th chapter.
\end{frame}

\begin{frame}[label=sec-2-3]{Church and society (culture)}
\emph{Christianity is radically not like society}
\alert{or}
\emph{Christianity is faith expressed through a society}

\begin{itemize}
\item The 19th c. wrestled with these statements, both of which seem to be true in some respects.
\end{itemize}
Where do we see them being reconciled in today's world?
\end{frame}

\begin{frame}[label=sec-2-4]{Revelation}
Between the statement ``The Bible is the very Word of God''
and the statement
``The Bible is an example of many kinds of literature, myth, preaching all mixed together but
very much a human product''
there is a great gulf.
What would a middle ground look like?
\end{frame}
% Emacs 24.3.1 (Org mode 8.2.3c)
\end{document}
