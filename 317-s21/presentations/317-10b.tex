\begin{document}

\maketitle
\begin{frame}{Outline}
\setcounter{tocdepth}{1}
\tableofcontents
\end{frame}

\section{Romantic era}
\label{sec-1}
\begin{frame}[label=sec-1-1]{Romantic period}
\begin{itemize}
\item appreciation of emotion and intuition
\item recovery and emphasis on national traditions (folklore)
\item over and against ``natural religion''
\end{itemize}
\end{frame}
\begin{frame}[label=sec-1-2]{Schleiermacher}
\begin{itemize}
\item persuade sophisticated friends not to reject religion
\item ``every event is a miracle'' ``every intuition and every original feeling'' is a revelation
\item religion based ``entirely on the inner experience of the believer''
\end{itemize}
\end{frame}
\begin{frame}[label=sec-1-3]{Coleridge}
\begin{itemize}
\item reject the tendency of 18th c. theology to base Christianity on argument
\item ``factual accuracy of the Bible does not much matter, for it can give us Truth in the way a great poem can''
\item religion is good for the nation?
\end{itemize}
\end{frame}
\section{Hegel and reaction to him}
\label{sec-2}
\begin{frame}[label=sec-2-1]{Hegel}
\begin{itemize}
\item importance of HX,
\item each epoch has its own genius
\item learn from ideas you don't like e.g. ``nationalism''
\item progress is messy
\item 3 moves: ``dialectic''
\item art has a purpose (no to ``art for art's sake''
\item institutions needed
\item growth will be painful
\end{itemize}
\end{frame}
\begin{frame}[label=sec-2-2]{``anti-hegelians''}
\begin{block}{Feuerbach}
\begin{itemize}
\item humans invent ``God''
\item trivial?
\end{itemize}
\end{block}
\begin{block}{Strauss}
\begin{itemize}
\item Gospels consist of ``myths''
\item ``to treat the Gospels as historically accurate would be to take metaphors and images as literal truth.''
\end{itemize}
\end{block}
\begin{block}{Marx}
\begin{itemize}
\item the philosophers have only \emph{interpreted} the world, in various ways; the point, however, is to \emph{change it}.
\end{itemize}
\end{block}
\end{frame}
\begin{frame}[label=sec-2-3]{S. Kierkegaard}
\begin{itemize}
\item furious production of books
\item \emph{Either/Or} \& \emph{Fear and Trebling}: give up our sentimental notions
\item enemies: the smug, bourgeousie
\item new ideas re. ``love'' that he mocked
\item ``impossible choices''
\item Life lived forward but understood backwards
\item ``leap of faith'' in faith
\item 3 approaches to religion
\begin{itemize}
\item aesthetic (beautiful)
\item ethical (rational)
\item religious (absurd)
\end{itemize}
\end{itemize}
\end{frame}
\section{State religion}
\label{sec-3}
\begin{frame}[label=sec-3-1]{Roman Catholic}
\begin{itemize}
\item 19th c. appreciation of middle ages
\item Napoleon's treaty with the Pope -- compromising the state? or the church?
\item Piux IX proclaiming \emph{Immaculate Conception} -- Pope ``infallible''
\item Leo XIII speaking on society sounds like a socialist
\item but reacting to the age -- he condemned ``modernists''
\end{itemize}
\end{frame}
\begin{frame}[label=sec-3-2]{Church of England}
\begin{itemize}
\item had a set of virtues and faults opposite to those of Rome
\item ``established'' church
\item Newman: England would improve ``were it vastly more superstitious, more bigoted, more gloomy, more fierce in its religion, than at present it shows itself to be.''
\end{itemize}
\end{frame}
\begin{frame}[label=sec-3-3]{Newman}
\begin{itemize}
\item mistrust state control -- but also ``liberal'' sense that one could be indifferent to doctrine
\item ``development'' of doctrine
\item many truths we cannot prove (England an island?)
\end{itemize}
\end{frame}
\section{Moving toward the 20th c.}
\label{sec-4}
\begin{frame}[label=sec-4-1]{Reactions}
\begin{itemize}
\item Maurice: rejecting both liberal and conservative, rational and romantic
\item Darwin: raising the question ``how to understand God's action in history in light of modern science
\item Ritschl, Harnack \& Troeltsch
\item Valuing science and history and \alert{also} maintaining the ``truth'' of Christianity
\end{itemize}
\end{frame}
\begin{frame}[label=sec-4-2]{Schweitzer}
\begin{itemize}
\item Renaissance man: physician, musician, ``saint''?, major theologian
\item Jesus was fundamentally an apocalyptic itinerant preacher
\item ``Quest for the historical Jesus'' continues to this day
\end{itemize}
\end{frame}
\begin{frame}[label=sec-4-3]{Nietzsche}
\begin{itemize}
\item reject the weakness, lowliness, and pity of Christianity
\item reject the notion that history is ``advancing'' -- ``truth'' is simply whatever a society agrees to call ``true''
\end{itemize}
\begin{block}{fascination with history}
that had seemed to hold such promise at the beginning of the century seemed to end in despair
\end{block}
\end{frame}
% Emacs 24.3.1 (Org mode 8.2.10)
\end{document}
