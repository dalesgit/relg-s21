\begin{document}

\maketitle
\begin{frame}{Outline}
\setcounter{tocdepth}{1}
\tableofcontents
\end{frame}

\begin{center}

\end{center}
\section{Overview}
\label{sec-1}
\begin{frame}[label=sec-1-1]{questions/focus}
\begin{itemize}
\item ``reform'' as in ``reformed life''
\item organizing society, community
\item what is community?
\item 39 articles
\item ``Puritans''
\end{itemize}
\end{frame}

\section{Calvinism}
\label{sec-2}
\begin{frame}[label=sec-2-1]{Calvinism}
\begin{block}{Discipline of the community}
\begin{itemize}
\item (191) Community maintain discipline by regulating who come to communion
\item (191) organization of community (Geneva) ministers, teachers, elders, deacons \ldots{} ``presbytery''
\end{itemize}
\end{block}
\end{frame}

\begin{frame}[label=sec-2-2]{Calvinism}
\begin{block}{Church and State}
\begin{itemize}
\item (192) Calvin allowing for working against civil authority when they betray the faith
\item John Knox in Scotland ``seized'' on those principles in his energy opposing the English
\item (193) France, Scotland, then England Calvinists thought about how to oppose an unjust society
\item Scotland corrupt church 193, thus as in France political conditions drove Calvin's followers
\end{itemize}
\end{block}
\end{frame}

\begin{frame}[label=sec-2-3]{Calvinism}
\begin{block}{``Orthodoxy''}
\begin{itemize}
\item in Amsterdam with Arminius -- in move against him, the church solidified its ``orthodoxy'' and \alert{thus froze positions that earlier had been nuanced}
\item p. 194 table of \alert{5 basic principles of Calvinist orthodoxy}
\item reading Calvin elicits not an emotional response but a cumulative one from the systematic presentation p 188
\end{itemize}
\end{block}
\end{frame}

\begin{frame}[label=sec-2-4]{Calvinism}
\begin{block}{now reform of the person not of the church}
\begin{itemize}
\item response to (free) grace is a \alert{reformed} life (thus the name)
\end{itemize}
\end{block}

\begin{block}{Predestination}
\begin{itemize}
\item clearly we are not saved by good works, therefore it has to be God's decision
\item ``Reflections on how we come to be saved led to the doctrine of predestination (189 ff.)
\item single? double?
\item \url{https://en.wikipedia.org/wiki/The_Private_Memoirs_and_Confessions_of_a_Justified_Sinner}
\end{itemize}
\end{block}
\end{frame}
\begin{frame}[label=sec-2-5]{epithoughts}
\begin{itemize}
\item ``Calvinist in polity'' -- huge influence on English world
\item Knox and Calvin and the ``reformed'' tradition
\end{itemize}
\end{frame}
\section{Beyond Geneva}
\label{sec-3}
\begin{frame}[label=sec-3-1]{Bucer}
\begin{itemize}
\item centrale figure of Reformation? 190
\item sought to avoid being explicit about sacraments (Luth/Zwingli)
\item stands out in the figures for his tolerance
\item theology of sacraments: cf. \alert{Martin Bucer} (large influence)
\begin{itemize}
\item sought position between Luther and Zwingli
\end{itemize}
\end{itemize}
\end{frame}

\begin{frame}[label=sec-3-2]{Knox}
\begin{itemize}
\item connected the struggle of Scotland against England with Reformed theology
\item what is the role of faithful viz a viz oppressive society?
\item synod of Dort (5 principles) ``uncompromising'' 194
\end{itemize}
\end{frame}

\begin{frame}[label=sec-3-3]{England}
\begin{itemize}
\item author disputes that Henry VIII ``started'' Anglican church
\item tradition of reform went back a century or more
\item Thomas Cranmer as Archbishop was the prime mover --
\end{itemize}
\end{frame}

\begin{frame}[label=sec-3-4]{}
\begin{itemize}
\item particularly in focus on worship and \emph{The Book of Common Prayer} (together with the later emerging \emph{King James Bible})
\item \emph{Lex orandi, lex credendi} is a fundamental character of Anglicanism (relation between worship and belief)

\item ``Anglican (195) could hold any theology from near Catholic to Calvinist''
\end{itemize}
\end{frame}
\begin{frame}[label=sec-3-5]{From Puritans to Quakers}
\begin{itemize}
\item Puritans to ``purify'' the church -- particularly with regard to worship (only scripture)
\item ``purifying'' church, worked hard and saved their money
\item as ``character'' working hard (capitalism, spirit of U.S.) -- cf. because of impact on US
\item Puritans and Capitalism (Weber)
\end{itemize}
\end{frame}

\begin{frame}[label=sec-3-6]{}
\begin{itemize}
\item Anglicans who reacted to Puritans: John Donne, William Laud, Lancelot Andrewes, \alert{Richard Hooker} and the \emph{via media}
\item (p. 197) ``Hooker did not accept the Roman Catholic position that tradition has an authority independent of Scripture, but he did use it as a reliable guide to the interpretation of Scripture, while the Puritans wanted to read their Bibles unencumbered by traditional assumptions.''
\item Oliver Cromwell and again the mixing of politics and religion

\item moderate and radical Puritans -- radical appealing to individual experience (of Spirit) cf. Quakers (George Fox)
\end{itemize}
\end{frame}
\begin{frame}[label=sec-3-7]{Compare Westminster \& 39 articles}
\end{frame}
% Emacs 24.3.1 (Org mode 8.2.3c)
\end{document}
