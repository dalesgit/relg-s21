\begin{document}

\maketitle
\begin{frame}{Outline}
\setcounter{tocdepth}{1}
\tableofcontents
\end{frame}

\section{Radical Reformation}
\label{sec-1}
\begin{frame}[label=sec-1-1]{The Schleitheim Confession Of Faith, 1527}
\begin{itemize}
\item a People set apart (sect)
\item the ``ban''
\item a ``remembrance'' for the holy people
\item the ``sword'' is for the world, not the holy people
\end{itemize}
\end{frame}

\section{Humanism}
\label{sec-2}
\begin{frame}[label=sec-2-1]{Erasmus}
Several quotes to illustrate ``Humanism''

\begin{itemize}
\item cf. \url{http://oll.libertyfund.org/people/desiderius-erasmus}
\item need for a philosopher of the kitchen
\item for peace and against war
\end{itemize}
\end{frame}

\section{Sacraments}
\label{sec-3}

\begin{frame}[label=sec-3-1]{3 Competing Definitions}
\begin{block}{Trent}
\begin{itemize}
\item Transubstantiation
\item Universal = Christ, outward appearance = bread and wine
\end{itemize}
\end{block}

\begin{block}{Luther}
\begin{itemize}
\item Consubstantiation
\item by the power of God both Christ and bread and wine present
\end{itemize}
\end{block}

\begin{block}{Zwingli}
\begin{itemize}
\item (mere) symbol
\item Humanist focus on the human effects, language
\end{itemize}
\end{block}
\end{frame}

\begin{frame}[label=sec-3-2]{Sacrament is \ldots{}}
``Outward and visible sign of an inward and invisible grace''
\begin{itemize}
\item one substance (Christ) with ``accidents'' (manifestations, ``outward appearance'') of bread and wine
\item But the power of God cannot be so determined and measured
\item In view of these passages we are compelled to confess that the words: ``This is my body,'' should not be understood naturally, but figuratively,
\item option: not to define the ``presence''
\end{itemize}
\end{frame}

\section{Counter Reformation}
\label{sec-4}
\begin{frame}[label=sec-4-1]{Council of Trent}
\begin{itemize}
\item p. 173 notion of ``justification'' which many thought as a kind of property people have or don't, -- L. ``no righteousness of our own but only share in Christ's righteousness''
\item Trent set forth doctrinal statements on Scripture and tradition, original sin, justification, and the sacraments that have provided the basis of Catholic theology ever since,'' laid foundation for reform
\end{itemize}
\end{frame}

\begin{frame}[label=sec-4-2]{}
\begin{itemize}
\item Scripture and tradition equally
\item L. had said original sin destroyed will, Catholics sought compromise
\item Justification as a ``process'' not instantaneous as L. thought
\item Sacraments and reform: 7 sacraments, transubstantiation, sacrifice on altar repeated, purgatory and indulgences (but warned re. abuse)
\end{itemize}
\end{frame}

\begin{frame}[label=sec-4-3]{Jesuits and mystics}
\begin{itemize}
\item Don Quixote and Loyola (Knight for Christ)
\item S.J. sending missionaries world wide, adapting to customs of people they met
\item Carmelites (John and Teresa)
\end{itemize}
\end{frame}
\begin{frame}[label=sec-4-4]{Continuing debates}
\begin{itemize}
\item Jesuits at center of response to Protestantism
\item ``Aquinas had taught that we talk about God ``analogically''
\item Port Royal Jansenists -- rigorous piety and trust in grace as opposed to Jesuits whom they saw as ``Pelagian''
\item Pascal: genius of age \ldots{} conversion ``Fire, God of Abraham \ldots{}'' trust in grace while being well trained in reason as mathematician
\item Pascal defense of Christianity, but ultimately Pope ruled against Jansenist radical trust in grace (not own will)
\item Fenelon, ``Quietism'', Mme. Guyon = claiming that usual rules don't apply because of special relationship with God
\end{itemize}
\end{frame}
% Emacs 24.3.1 (Org mode 8.2.3c)
\end{document}
